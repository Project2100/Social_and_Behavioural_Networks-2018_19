%!TEX root=../main.tex
\chapter{Network motifs}\label{sec:motifs}

Graphs representing networks, including biological networks and social networks, contain wide variety of subgraphs. One important local property of networks are so-called \emph{network motifs}, which are defined as recurrent and statistically significant sub-graphs or patterns.

Network motifs are sub-graphs that repeat themselves in a specific network or even among various networks. Each of these sub-graphs, defined by a particular pattern of interactions between vertices, may reflect a framework in which particular functions are achieved efficiently. Indeed, motifs are of notable importance largely because they may reflect functional properties.

Let's start by the simplest network motifs, the triangles.


\section{Triangles}\label{sec:triangles}

The triangle is, unsurprisingly, defined as a triple of nodes for which each of the three pairs of node is connected. 
We would like to compute the \emph{global clustering coefficient}:
\begin{equation}\label{eq:clustering-coefficient}
    c(G) = \frac{3 \cdot \left(\text{\# triangles in G}\right)}{\left(\text{\# connected triples in G}\right)} = \frac{3 \mathcal{T}}{w(G)}.
\end{equation}
%
Let's compute this on an example.
%
\begin{figure}[h!]
	\centering
	\includegraphics[width=.4\textwidth]{graph-triangle-example.png}
	\caption{Graph example}\label{fig:graph-example-triangles}
\end{figure}

In the graph shown in \cref{fig:graph-example-triangles} there are $5$ connected triples and $1$ triangle, so the global clustering coefficient would be $3/5$.
%
\begin{figure}[h!]
	\centering
	\includegraphics[width=\textwidth]{conn-triples.png}
\end{figure}

\obs The \emph{g.c.c.} for a clique is $1$, because every connected triple is also a triangle; while for a star it is $0$ because there are no triangles.

To compute the denominator for the \emph{g.c.c.}, that is $w(g)$, what we can do is sum all the pairs of nodes in the neighborhood of each node.
\[
	w(G) = \sum_{u \in V}\binom{deg(u)}{2}
\]
taking time $T = O(N)$.\\
This gives us a first algorithm to compute the \emph{g.c.c.}:
%
\begin{lstlisting}[caption={Algorithm 1}, label={lst:triangles-alg1}]
Algorithm_1:
    $\tau \gets 0$   // the number of triangles
    for $u \in V$:
        for $\{ v, w \} \in \binom{N_u}{2}$:   // where $N_u$ is the set of the neighbors of $u$
            if $\{ v, w \} \in E$:
                $\tau$ += 1
    return $\frac{\tau}{3}$
\end{lstlisting}
%
\begin{claim}\label{cl:triangles-1}
	Algorithm is in $O(n^3)$.
\end{claim}
\begin{proof}
    First, assume that we have both the adjacency matrix that represents the graph and the degree of each node.\\
    The complexity of the algorithm is given by $w(G)$, since we add at most 1 at each step, for each pair of nodes in $\binom{N_u}{2}$, thus,
	if the graph is dense -- that is the degree for each node $u$ is close to $n^2$ -- we have that:
	\[
		\sum_{u \in V}\binom{deg(u)}{2} = \sum_{u \in V}\Omega(n^2) = \Omega(n^3),
	\]
	while, if the \textit{max-degree} is constant, we have that:
	\[
		\sum_{u \in V}\binom{deg(u)}{2} = \sum_{u \in V} O(1) = O(n).
	\]
\end{proof}

Therefore, we would like to find other algorithms, or methods, to compute \textit{g.c.c.} faster, even for dense graphs.


\subsection{Adjacency Matrix method}\label{sec:triangles-adjacency}

One algorithm to compute the number of triangles is based on the adjacency matrix multiplication.
The adjacency matrix for the graph seen in \cref{fig:graph-example-triangles} is the following
\[
	A = \begin{pmatrix}
		0 & 1 & 1 & 0 \\
		1 & 0 & 1 & 0 \\
		1 & 1 & 0 & 1 \\
		0 & 0 & 1 & 0 \\
	\end{pmatrix}
\]
Let's compute the square of it.
\[
	A^2 = 	\begin{pmatrix}
		2 & 1 & 1 & 1 \\
		1 & 2 & 1 & 1 \\
		1 & 1 & 3 & 0 \\
		1 & 1 & 0 & 1 \\
	\end{pmatrix}
\]
Now, what does the $ij$ cell contain? By definition of matrix multiplication we have that
\[
	(A^2)_{uv} = \sum_{w \in G}A_{uw} \cdot A_{wv}
\]
where $A_{uw} = 1$ iff $(u,w) \in E$ and $A_{wv} = 1$ iff $(w,v) \in E$. In other words, we are counting the number of paths of length $2$ that go from $u$ to $w$.\\
In cell $A_{uu}$ there is the number of paths of length $2$ that go from $u$ to itself, that is equal to its degree because we can go from $u$ to any of its neighbours in $1$ step and then use the same edge to go back to $u$.\\
For $A^3$ we have that
\[
    (A^3)_{uv} = \left( A \cdot A^2 \right)_{uv} = \sum_{w \in G} A_{uw} \cdot (A^2)_{wv},
\]
thus, intuitively, $(A^3)_{uv}$ contains the number of paths of length $3$ that go from $u$ to $v$, and in general the same reasoning applies to any $A^k$ for paths of length $k$.

To find triangles we can compute $A^3$ and, for every node, get its diagonal entry in the matrix, effectively giving us the number of paths of length $3$ that go from that node to itself.

\begin{figure}[h!]
	\centering
	\includegraphics[width=.4\textwidth]{triangle.png}
	\caption{A triangle.}\label{fig:triangle-graph}
\end{figure}

Let's compute $(A^3)_{uu}$ in \cref{fig:triangle-graph}; there are two paths of length $3$ that go from $u$ to itself: $uvw, uwv$. So $(A^3)_{uu}$ would be $2$, but the node is in just one triangle.
That is true in general, so every triangle gives a contribute of $2$ to the diagonal entry of every node that forms it. Furthermore, this way we will count the same triangle $3$ times (1 for each vertex of the triangle).

\begin{defn}[Trace of a matrix]
    Let $m$ be a matrix, its trace is
    \begin{equation}\label{eq:trace}
        tr(M) = \sum_i M_{ii}.
    \end{equation}
\end{defn}

We thus have
\[
	\sum_{u \in G}{\left(A^3\right)_{uu}} = tr(A^3) = 6 \cdot \tau(G).
\]

\begin{thm}\label{thm:triangles-1}
    The number of triangles $\tau$ of a graph $G$ is given by
	$$\tau(G) = \frac{tr(A^3)}{6}.$$
\end{thm}

This gives us the following algorithm, that uses an algebraic approach based on the adjacency matrix it takes in input:
\begin{lstlisting}[caption={Algorithm 2}, label={lst:triangles-alg2}]
Algorithm_2(A):
    $AC \gets$ fast_mat_multiplication(A,3)
    return $tr(AC)/6$
\end{lstlisting}

\obs In general, to compute $A^i$, you need $\log_2(i)$ matrix multiplications.

\obs The execution time is given by the matrix multiplication, for which the best known algorithm takes time $O(n^{2.37})$, where $n$ is the number of nodes. Therefore, the complexity of [\ref{lst:triangles-alg2}] is independent from the number of edges, that is, the algorithm doesn't take into account the structure of the graph.\\
On the one hand this approach gives an improvement for dense graphs, for which we pass from $O(n^3)$ to $O(n^{2.37})$, while on the other hand we can get a worsening for sparse graphs, such as for the star, for which we pass from $O(n^2)$ to $O(n^{2.37})$.\\
Note that this is true because for a sparse matrix $A$, its square $A^2$ isn't sparse; consider the $A^2$ for a star, it is dense since there are a lot of possible two-paths.


\subsection{Structure-based method}\label{sec:triangles-structure}

Let's consider the star again, in algorithm [\ref{lst:triangles-alg1}] we loose a lot of time by making operations on the root node, that are useless, since we find the same ``potential triangles'' starting from the leaves.

\textit{Idea}: If we sort the nodes by their degree, and we follow this ordering when we look for triangles, we won't count the triangles more times, and we'll have less operations to do when we reach the nodes with higher degree (since we'll have already considered most of their neighbors).

\begin{ex}
    The ordering of the nodes of graph in figure [\ref{fig:graph-example-triangles}] is: \textit{4, 1, 2, 3}, since the degrees of the nodes are, respectively, \textit{1, 2, 2, 3}.
\end{ex}

This gives us the following algorithm:
\begin{lstlisting}[caption={Algorithm 2}, label={lst:triangles-alg3}]
Algorithm_3(G):
    $\tau \gets 0$
    $order(V) \gets$ sort $V$ so that $u < v \Longleftrightarrow  deg(u) \leq deg(v)$
    for each $u \in order(V)$:
        $N_u^+ \gets \{ v \in N_u \st v > u \}$
        for every $\{ v,w \} \in \binom{N_u^+}{2}$:
            if $\{ v,w \} \in E$:
                $\tau$ += 1
\end{lstlisting}

\begin{thm}\label{thm:triangles-2}
    \texttt{Algorithm\_3} [\ref{lst:triangles-alg3}] returns exactly $\tau(G)$.
\end{thm}
\begin{proof}
    Every triangle is counted exactly once by $w = \min \{ x \in triangle \}$, that is, the vertex with minimum degree in the triangle.
\end{proof}

\begin{thm}\label{thm:triangles-3}
    Let $\mathcal{E} = |E|$, \texttt{Algorithm\_3} [\ref{lst:triangles-alg3}] runs in $O(\mathcal{E}^{3/2})$.
\end{thm}

\ex Let's consider the star again, with Algorithm\_3 we need only $O(n^{1.5})$, instead of $O(n^2)$.

\obs Since we add at most 1 triangle at each step, there will be at most $O(\mathcal{E}^{3/2})$ triangles in the graph, furthermore \texttt{Algorithm\_3} [\ref{lst:triangles-alg3}] is able to return the whole list of triangles, along with their number.

\ex In a clique there are $\mathcal{E}^{3/2}$ triangles: $\mathcal{E} = \Theta(n^2),\ \tau(G) = \Theta(n^3) \Longrightarrow \tau(G) = \Theta(\mathcal{E}^{3/2})$. Then we cant do better than this in the worst case, but in general social graphs have less edges and so we can obtain better performances.

\begin{proof}[Proof of theorem \ref{thm:triangles-3}]
    We prove the theorem putting together some observations and lemmas.
    
    \begin{obs}\label{obs:triangles-1}
        First of all, consider that $\mathcal{E} = \frac12 \sum_{u \in G} deg(u)$, thus there are at most $2 \sqrt{\mathcal{E}}$ nodes with degree $\geq \sqrt{\mathcal{E}}$.
    \end{obs}
    
    Let $V_1 = \{ v \in V \st deg(v) \geq \sqrt{\mathcal{E}} \}$ and $V_2 = V - V_1 = \{ v \in V \st deg(v) < \sqrt{\mathcal{E}} \}$.\\
    By observation [\ref{obs:triangles-1}] we know that $|V_1| \leq 2 \sqrt{\mathcal{E}}$, then $E \geq \frac12 \sum_{u \in V_1} \sqrt{\mathcal{E}} = \frac12 |V_1| \sqrt{\mathcal{E}}$.
    
    \begin{obs}\label{obs:triangles-2}
        \begin{equation}\label{eq:triangles-t}
            T = \sum_{u \in V_2} \abs{\binom{N_u^+}{2}} \leq \sum_{u \in V} \abs{N_u^+}^2.
        \end{equation}
    \end{obs} % todo: is it V or V_1 ???

    \begin{lem}\label{lem:triangles-1}
        $\abs{N_u^+} \leq 2 \sqrt{\mathcal{E}}\ \forall\ u \in V_1$.
    \end{lem}
    \begin{proof}
        Refer to figure [\ref{fig:triangles-l1}].
        
        \begin{figure}[h!]
            \centering
            \includegraphics[width=.4\textwidth]{triangles_l1.png}
            \caption{A visual proof of lemma [\ref{lem:triangles-1}]}
            \label{fig:triangles-l1}
        \end{figure}
    \end{proof}

    Then we can define $T_{V_1}$ and give it an upper bound (this will allow us to prove an upper bound for $T$, together with $T_{V_2}$, that we'll define later):
    \begin{equation}\label{eq:triangles-tv1}
        T_{V_1} = \sum_{u \in V_1} \abs{\binom{N_u^+}{2}} \leq \sum_{u \in V_1} \abs{N_u^+}^2 \leq 8 \mathcal{E}^{3/2}.
    \end{equation}
    
    In order to give a similar bound to $T_{V_2}$, we partition the nodes in $V_2$ in buckets as follows, according with their degree:\\
    $V_2 = V_2^{(1)} \cup V_2^{(2)} \cup \ldots \cup V_2^{(i)} \cup \ldots \cup V_2^{(k)}$, where $V_2^{(1)} = \{ v \st v \in V \wedge \frac{\sqrt{\mathcal{E}}}{2} \leq deg(v) < \sqrt{\mathcal{E}} \}$, $V_2^{(2)} = \{ v \st v \in V \wedge \frac{\sqrt{\mathcal{E}}}{4} \leq deg(v) < \frac{\sqrt{\mathcal{E}}}{2} \}$, $V_2^{(i)} = \{ v \st v \in V \wedge \frac{\sqrt{\mathcal{E}}}{2^{i+1}} \leq deg(v) < \frac{\sqrt{\mathcal{E}}}{2^i} \}$, $V_2^{(k)} = \{ v \st v \in V \wedge 1 \leq deg(v) < 2 \}$.\\
    We know that $\abs{V_2^{(1)}} \leq 4 \sqrt{\mathcal{E}}$ by lemma [\ref{lem:triangles-1}], then we can compute $\abs{V_2^{(2)}} \leq \frac{2 \mathcal{E}}{\frac{\sqrt{\mathcal{E}}}{4}} = 8 \sqrt{\mathcal{E}}$, $\abs{V_2^{(i)}} \leq \frac{2 \mathcal{E}}{\frac{\sqrt{\mathcal{E}}}{2^{i+1}}} = 2^{i+1} \sqrt{\mathcal{E}}$.
    
    We can underline our last result in the following lemma:
    \begin{lem}\label{lem:triangles-2}
        \begin{equation}
            \abs{V_2^{(i)}} \leq 2^{i+1} \sqrt{\mathcal{E}}.
        \end{equation}
    \end{lem}

    \begin{lem}\label{lem:triangles-3}
        \begin{equation}\label{eq:triangles-tv2}
            T_{V_2} = \sum_{u \in V_2} \abs{\binom{N_u^+}{2}} \leq \sum_{u \in V_2} \abs{N_u^+}^2 \leq \sum_{u \in V_2} deg(u)^2.
        \end{equation}
    \end{lem}

    \begin{lem}\label{lem:triangles-4}
        For $u \in V_2^{(i)}$, \texttt{algorithm-3} [\ref{lst:triangles-alg3}] does at most $\mathcal{E} 2^{-2i}$ steps.
    \end{lem}

    Thus, we can conclude the proof of theorem by giving an upper bound for $T_{V_2}$, that -- together with the bound for $T_{V_1}$ shown in [\ref{eq:triangles-tv1}] -- will give us a bound for $T$ [\ref{eq:triangles-t}]:
    \begin{flalign*}
        T_{V_2} &= T_{V_2^{(1)}} + T_{V_2^{(2)}} + \ldots + T_{V_2^{(i)}} + \ldots + T_{V_2^{(k)}}&\\
        &= \sum_{i=1}^{\log_2 \sqrt{\mathcal{E}}} \left( \abs{V_2^{(i)}} \cdot \left(\text{ work for } u \in V_2^{(i)} \right) \right)&\\
        &= \sum_{i=1}^{\log_2 \sqrt{\mathcal{E}}} \left( 2^{i+2} \sqrt{\mathcal{E}} \cdot \mathcal{E}2^{-2i} \right)&\\
        &\leq \mathcal{E} \sqrt{\mathcal{E}} \cdot \sum_{i=1}^{\log_2 \sqrt{\mathcal{E}}} 2^{-2i} \leq 4 \mathcal{E}^{3/2}.&
    \end{flalign*}
\end{proof}
